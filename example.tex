\documentclass[aspectratio=169]{beamer}

% Use the Shibui theme - Japanese aesthetic of simple, subtle beauty
% Options:
%   - Color: light (default), dark, solarizedlight, solarizeddark, solarizedosaka,
%            nord, gruvboxlight, gruvboxdark, autumn, inei, ineienhanced, ineilight
%   - Font: sans (default), serif, garamond, charter, mono
%   - Layout: nosectionheader (default, hide section from header), navontop (show navigation above title)
%   - TOC: nosub (hide subsections in table of contents)
%   - Progress bar: progressbar=basic (default), progressbar=segmented, progressbar=none
%
% Examples:
%   \usetheme{Shibui}                                   % Use all defaults
%   \usetheme[solarizedosaka]{Shibui}                   % Just color scheme
%   \usetheme[nord, serif]{Shibui}                      % Color + font
%   \usetheme[dark, navontop]{Shibui}                   % Show navigation squares above title
%   \usetheme[gruvboxdark, garamond]{Shibui}            % Multiple options
%   \usetheme[nord, progressbar=segmented]{Shibui}      % Section-based progress bar
%   \usetheme[inei, garamond]{Shibui}                   % In Praise of Shadows aesthetic with textured background
\usetheme[ineienhanced, charter, nosub, progressbar=segmented, kanjilogo]{Shibui}
% \renewcommand{\ineilightcolumns}{5}  % Deprecated: now uses subtle vignetting instead

% Custom footer text (optional)
\footertext{Conference Name 2024 -- City, Country}

% Additional packages for examples
\usepackage{CJKutf8}  % For Japanese kanji in title
\usepackage{listings}
\usepackage{algorithm}
\usepackage{algpseudocode}
\usepackage{pgfplots}
\pgfplotsset{compat=1.18}

% Helper command for color preview boxes
\newcommand{\colorpreview}[2]{%
  \tikz\node[rectangle, fill=#1, minimum width=1.2cm, minimum height=0.5cm,
             draw=shibuitext, line width=0.3pt, inner sep=0pt]{};%
  \,\texttt{\footnotesize #2}%
}

% NOTE: Listings and algorithm styling are now handled automatically by the theme!
% You can still override settings here if needed.

% Metadata
\title{The Shibui \begin{CJK}{UTF8}{min}(渋い)\end{CJK} Beamer Theme}
\subtitle{A Minimalist Design for Clear Communication}
\author{Your Name}
\institute{Your Institution}
\date{\today}

\begin{document}

% Title page (automatically formatted, no header/footer)
\maketitle

% Table of contents
\begin{frame}{Overview}
  \tableofcontents
\end{frame}

%--------------------------------------------------
\section{Introduction}
%--------------------------------------------------

% Optional: uncomment to show section slide
% \sectionframe

\begin{frame}{Design Philosophy}
  The Shibui theme embodies Japanese minimalist design principles:

  \begin{itemize}
    \item Clarity through simplicity
    \item Generous use of whitespace
    \item Elegant typography
    \item Minimal but effective navigation
    \item Focus on content, not decoration
  \end{itemize}

  \vspace{1em}
  Shibui represents simple, subtle, and unobtrusive beauty through refined restraint.
\end{frame}

\begin{frame}{Key Features}
  \begin{enumerate}
    \item \textbf{Multiple Color Schemes}: 9 themes from light to dark, including Solarized, Nord, Gruvbox
    \item \textbf{Typography Options}: 5 font families with matching math fonts
    \item \textbf{Progress Indicators}: Basic, segmented, or none
    \item \textbf{Minimal Navigation}: Clean headlines with optional section squares
    \item \textbf{Academic Features}: Theorem environments, proofs, citations
  \end{enumerate}
\end{frame}

\begin{frame}[fragile, allowframebreaks]{Quick Start}
  \textbf{Basic usage:}

  \vspace{0.5em}

  \begin{shibuiframe}{Minimal Example}
    \texttt{\textbackslash documentclass\{beamer\}} \\
    \texttt{\textbackslash usetheme\{Shibui\}} \\
    \texttt{\textbackslash begin\{document\}} \\
    \texttt{...your content...} \\
    \texttt{\textbackslash end\{document\}}
  \end{shibuiframe}

  \vspace{1em}

  \textbf{With options:}

  \vspace{0.5em}

  \begin{shibuiframe}{Customized Theme}
    \texttt{\textbackslash usetheme[nord, charter]\{Shibui\}} \\
    \texttt{\textbackslash usetheme[dark, progressbar=segmented]\{Shibui\}} \\
    \texttt{\textbackslash usetheme[solarizedosaka, garamond, kanjilogo]\{Shibui\}}
  \end{shibuiframe}

  \vspace{0.5em}

  \small See following sections for detailed customization options.
\end{frame}

%--------------------------------------------------
\section{Theme Options}
%--------------------------------------------------

\subsection{Typography}

\begin{frame}{Typography Overview}
  Good typography is essential for effective presentations.

  \vspace{0.5em}

  \textbf{This theme offers 5 font options with matching math fonts:}

  \begin{itemize}
    \item \textbf{sans}: Fira Sans + sfmath (sans-serif math)
    \item \textbf{serif}: ET Book/Palatino + mathpazo (Palatino math)
    \item \textbf{garamond}: EB Garamond + newtxmath (Times-like math)
    \item \textbf{charter}: Charter + mathdesign charter (matching math)
    \item \textbf{mono}: Fira Mono + Computer Modern math
  \end{itemize}

  \vspace{0.5em}

  \textbf{Math font override options:} \texttt{mathdefault} (auto-match text), \texttt{mathsans}, \texttt{mathserif}

  \vspace{0.5em}

  \small \emph{Example:} \texttt{\textbackslash usetheme[garamond, mathsans]\{Shibui\}} uses Garamond text with sans-serif math.
\end{frame}

\begin{frame}{Font: sans (Fira Sans + sfmath)}
  \textbf{Text:} Fira Sans \quad \textbf{Math:} sfmath (sans-serif)

  \vspace{0.5em}

  \textbf{Usage:} \texttt{\textbackslash usetheme[sans]\{Shibui\}} or \texttt{\textbackslash usetheme\{Shibui\}} (default)

  \vspace{1em}

  {\fontfamily{FiraSans-TLF}\selectfont
  The Fibonacci sequence is defined by $F_n = F_{n-1} + F_{n-2}$ with $F_0 = 0$ and $F_1 = 1$.

  For continuous functions, the integral $\int_a^b f(x)\,dx$ represents the area under the curve.

  Einstein's mass-energy equivalence: $E = mc^2$}
\end{frame}

\begin{frame}{Font: serif (ET Book/Palatino + mathpazo)}
  \textbf{Text:} ET Book/Palatino \quad \textbf{Math:} mathpazo (Palatino math)

  \vspace{0.5em}

  \textbf{Usage:} \texttt{\textbackslash usetheme[serif]\{Shibui\}}

  \vspace{1em}

  {\fontfamily{ETbb-TLF}\selectfont
  The Fibonacci sequence is defined by $F_n = F_{n-1} + F_{n-2}$ with $F_0 = 0$ and $F_1 = 1$.

  For continuous functions, the integral $\int_a^b f(x)\,dx$ represents the area under the curve.

  Einstein's mass-energy equivalence: $E = mc^2$}
\end{frame}

\begin{frame}{Font: garamond (EB Garamond + newtxmath)}
  \textbf{Text:} EB Garamond \quad \textbf{Math:} newtxmath (Times-like)

  \vspace{0.5em}

  \textbf{Usage:} \texttt{\textbackslash usetheme[garamond]\{Shibui\}}

  \vspace{1em}

  {\fontfamily{EBGaramond-TLF}\selectfont
  The Fibonacci sequence is defined by $F_n = F_{n-1} + F_{n-2}$ with $F_0 = 0$ and $F_1 = 1$.

  For continuous functions, the integral $\int_a^b f(x)\,dx$ represents the area under the curve.

  Einstein's mass-energy equivalence: $E = mc^2$}
\end{frame}

\begin{frame}{Font: charter (Charter + mathdesign)}
  \textbf{Text:} Charter \quad \textbf{Math:} mathdesign charter (matching)

  \vspace{0.5em}

  \textbf{Usage:} \texttt{\textbackslash usetheme[charter]\{Shibui\}}

  \vspace{1em}

  {\fontfamily{bch}\selectfont
  The Fibonacci sequence is defined by $F_n = F_{n-1} + F_{n-2}$ with $F_0 = 0$ and $F_1 = 1$.

  For continuous functions, the integral $\int_a^b f(x)\,dx$ represents the area under the curve.

  Einstein's mass-energy equivalence: $E = mc^2$}
\end{frame}

\begin{frame}{Font: mono (Fira Mono + CM math)}
  \textbf{Text:} Fira Mono \quad \textbf{Math:} Computer Modern (default)

  \vspace{0.5em}

  \textbf{Usage:} \texttt{\textbackslash usetheme[mono]\{Shibui\}}

  \vspace{1em}

  {\fontfamily{FiraMono-TLF}\selectfont
  The Fibonacci sequence is defined by $F_n = F_{n-1} + F_{n-2}$ with $F_0 = 0$ and $F_1 = 1$.

  For continuous functions, the integral $\int_a^b f(x)\,dx$ represents the area under the curve.

  Einstein's mass-energy equivalence: $E = mc^2$}
\end{frame}

\begin{frame}[fragile]{Math Font Options}
  The theme provides flexible math font configuration:

  \vspace{0.5em}

  \begin{shibuiframe}{Default Behavior (mathdefault)}
    \texttt{\textbackslash usetheme[sans]\{Shibui\}} → sfmath (sans-serif) \\
    \texttt{\textbackslash usetheme[serif]\{Shibui\}} → mathpazo (Palatino) \\
    \texttt{\textbackslash usetheme[garamond]\{Shibui\}} → newtxmath (Times-like) \\
    \texttt{\textbackslash usetheme[charter]\{Shibui\}} → mathdesign (Charter) \\
    \texttt{\textbackslash usetheme[mono]\{Shibui\}} → Computer Modern
  \end{shibuiframe}

  \vspace{0.5em}

  \begin{shibuiframe}{Override Math Font}
    \textbf{Force sans-serif math:} \\
    \texttt{\textbackslash usetheme[garamond, mathsans]\{Shibui\}} \\
    \vspace{0.3em}
    \textbf{Force serif math:} \\
    \texttt{\textbackslash usetheme[sans, mathserif]\{Shibui\}}
  \end{shibuiframe}

  \vspace{0.5em}

  \small Mix and match text and math fonts for maximum flexibility.
\end{frame}

\subsection{Color Schemes}

\begin{frame}{Color Palette Overview}
  The theme now supports \textbf{12 color schemes} for different aesthetics.

  \vspace{0.5em}

  \textbf{Available schemes:} light, dark, solarizedlight, solarizeddark, solarizedosaka, nord, gruvboxlight, gruvboxdark, autumn, inei, ineienhanced, ineilight

  \vspace{0.5em}

  Each scheme uses these color roles:
  \begin{itemize}
    \item \texttt{bg} -- background
    \item \texttt{text} -- main text
    \item \texttt{accent} -- highlights
    \item \texttt{gray} -- UI elements
    \item \texttt{darkgray} -- secondary text
    \item \texttt{codebg} -- code background
  \end{itemize}

  \vspace{0.5em}

  The following slides show each color scheme with preview boxes.
\end{frame}

\begin{frame}{Color Scheme: light}
  \begin{shibuiframe}{light -- Warm cream with dark red accent}
    \colorpreview{shibuibg@light}{bg}
    \colorpreview{shibuitext@light}{text}
    \colorpreview{shibuiaccent@light}{accent}
    \colorpreview{shibuigray@light}{gray}
    \colorpreview{shibuidarkgray@light}{darkgray}
    \colorpreview{shibuicodebg@light}{codebg}
  \end{shibuiframe}

  \vspace{1em}

  Classic Shibui aesthetic with warm cream background and sophisticated dark red accents. Perfect for traditional presentations.

  \vspace{0.5em}

  \textbf{Usage:} \texttt{\textbackslash usetheme[light]\{Shibui\}} or \texttt{\textbackslash usetheme\{Shibui\}} (default)
\end{frame}

\begin{frame}{Color Scheme: dark}
  \begin{shibuiframe}{dark -- Dark background with red accent}
    \colorpreview{shibuibg@dark}{bg}
    \colorpreview{shibuitext@dark}{text}
    \colorpreview{shibuiaccent@dark}{accent}
    \colorpreview{shibuigray@dark}{gray}
    \colorpreview{shibuidarkgray@dark}{darkgray}
    \colorpreview{shibuicodebg@dark}{codebg}
  \end{shibuiframe}

  \vspace{1em}

  Modern dark theme ideal for low-light presentations. Reduces eye strain during long sessions.

  \vspace{0.5em}

  \textbf{Usage:} \texttt{\textbackslash usetheme[dark]\{Shibui\}}
\end{frame}

\begin{frame}{Color Scheme: solarizedlight}
  \begin{shibuiframe}{solarizedlight -- Ethan Schoonover's light palette}
    \colorpreview{shibuibg@solarizedlight}{bg}
    \colorpreview{shibuitext@solarizedlight}{text}
    \colorpreview{shibuiaccent@solarizedlight}{accent}
    \colorpreview{shibuigray@solarizedlight}{gray}
    \colorpreview{shibuidarkgray@solarizedlight}{darkgray}
    \colorpreview{shibuicodebg@solarizedlight}{codebg}
  \end{shibuiframe}

  \vspace{1em}

  Carefully designed by Ethan Schoonover for optimal readability. Selective contrast reduces eye fatigue.

  \vspace{0.5em}

  \textbf{Usage:} \texttt{\textbackslash usetheme[solarizedlight]\{Shibui\}}
\end{frame}

\begin{frame}{Color Scheme: solarizeddark}
  \begin{shibuiframe}{solarizeddark -- Ethan Schoonover's dark palette}
    \colorpreview{shibuibg@solarizeddark}{bg}
    \colorpreview{shibuitext@solarizeddark}{text}
    \colorpreview{shibuiaccent@solarizeddark}{accent}
    \colorpreview{shibuigray@solarizeddark}{gray}
    \colorpreview{shibuidarkgray@solarizeddark}{darkgray}
    \colorpreview{shibuicodebg@solarizeddark}{codebg}
  \end{shibuiframe}

  \vspace{1em}

  Dark variant of Solarized with blue accents. Scientifically designed color relationships.

  \vspace{0.5em}

  \textbf{Usage:} \texttt{\textbackslash usetheme[solarizeddark]\{Shibui\}}
\end{frame}

\begin{frame}{Color Scheme: solarizedosaka}
  \begin{shibuiframe}{solarizedosaka -- Deep blue with muted red}
    \colorpreview{shibuibg@solarizedosaka}{bg}
    \colorpreview{shibuitext@solarizedosaka}{text}
    \colorpreview{shibuiaccent@solarizedosaka}{accent}
    \colorpreview{shibuigray@solarizedosaka}{gray}
    \colorpreview{shibuidarkgray@solarizedosaka}{darkgray}
    \colorpreview{shibuicodebg@solarizedosaka}{codebg}
  \end{shibuiframe}

  \vspace{1em}

  Custom Solarized variant with deep blue-cyan background and muted red accents. Combines elegance with readability.

  \vspace{0.5em}

  \textbf{Usage:} \texttt{\textbackslash usetheme[solarizedosaka]\{Shibui\}}
\end{frame}

\begin{frame}{Color Scheme: nord}
  \begin{shibuiframe}{nord -- Arctic, north-bluish palette}
    \colorpreview{shibuibg@nord}{bg}
    \colorpreview{shibuitext@nord}{text}
    \colorpreview{shibuiaccent@nord}{accent}
    \colorpreview{shibuigray@nord}{gray}
    \colorpreview{shibuidarkgray@nord}{darkgray}
    \colorpreview{shibuicodebg@nord}{codebg}
  \end{shibuiframe}

  \vspace{1em}

  Inspired by Arctic beauty with cool blue tones. Clean and calm aesthetic for modern presentations.

  \vspace{0.5em}

  \textbf{Usage:} \texttt{\textbackslash usetheme[nord]\{Shibui\}}
\end{frame}

\begin{frame}{Color Scheme: gruvboxlight}
  \begin{shibuiframe}{gruvboxlight -- Retro groove with warm tones}
    \colorpreview{shibuibg@gruvboxlight}{bg}
    \colorpreview{shibuitext@gruvboxlight}{text}
    \colorpreview{shibuiaccent@gruvboxlight}{accent}
    \colorpreview{shibuigray@gruvboxlight}{gray}
    \colorpreview{shibuidarkgray@gruvboxlight}{darkgray}
    \colorpreview{shibuicodebg@gruvboxlight}{codebg}
  \end{shibuiframe}

  \vspace{1em}

  Retro-inspired warm color palette. Bright orange accents with earthy tones create an inviting atmosphere.

  \vspace{0.5em}

  \textbf{Usage:} \texttt{\textbackslash usetheme[gruvboxlight]\{Shibui\}}
\end{frame}

\begin{frame}{Color Scheme: gruvboxdark}
  \begin{shibuiframe}{gruvboxdark -- Dark retro with bright orange}
    \colorpreview{shibuibg@gruvboxdark}{bg}
    \colorpreview{shibuitext@gruvboxdark}{text}
    \colorpreview{shibuiaccent@gruvboxdark}{accent}
    \colorpreview{shibuigray@gruvboxdark}{gray}
    \colorpreview{shibuidarkgray@gruvboxdark}{darkgray}
    \colorpreview{shibuicodebg@gruvboxdark}{codebg}
  \end{shibuiframe}

  \vspace{1em}

  Dark variant with vibrant orange accents. Combines nostalgia with modern design sensibilities.

  \vspace{0.5em}

  \textbf{Usage:} \texttt{\textbackslash usetheme[gruvboxdark]\{Shibui\}}
\end{frame}

\begin{frame}{Color Scheme: autumn}
  \begin{shibuiframe}{autumn -- Warm earthy tones}
    \colorpreview{shibuibg@autumn}{bg}
    \colorpreview{shibuitext@autumn}{text}
    \colorpreview{shibuiaccent@autumn}{accent}
    \colorpreview{shibuigray@autumn}{gray}
    \colorpreview{shibuidarkgray@autumn}{darkgray}
    \colorpreview{shibuicodebg@autumn}{codebg}
  \end{shibuiframe}

  \vspace{1em}

  Warm autumn palette with browns and rust orange. Creates a cozy, approachable presentation atmosphere.

  \vspace{0.5em}

  \textbf{Usage:} \texttt{\textbackslash usetheme[autumn]\{Shibui\}}
\end{frame}

\begin{frame}{Color Scheme: inei}
  \begin{shibuiframe}{inei -- In Praise of Shadows}
    \colorpreview{shibuibg@inei}{bg}
    \colorpreview{shibuitext@inei}{text}
    \colorpreview{shibuiaccent@inei}{accent}
    \colorpreview{shibuigray@inei}{gray}
    \colorpreview{shibuidarkgray@inei}{darkgray}
    \colorpreview{shibuicodebg@inei}{codebg}
  \end{shibuiframe}

  \vspace{1em}

  Inspired by Tanizaki's philosophy: beauty in dimness, depth through layered shadows. Deep sumi ink black like traditional Japanese books with cool muted tones for contemplative viewing.

  \vspace{0.5em}

  \textbf{Aesthetic:} Cool blue-gray text on near-black background, muted antique gold accents gleaming from darkness. Simple and fast with no background effects.

  \vspace{0.5em}

  \textbf{Usage:} \texttt{\textbackslash usetheme[inei]\{Shibui\}}
\end{frame}

\begin{frame}{Color Scheme: ineienhanced}
  \begin{shibuiframe}{ineienhanced -- In'ei with Effects}
    \colorpreview{shibuibg@ineienhanced}{bg}
    \colorpreview{shibuitext@ineienhanced}{text}
    \colorpreview{shibuiaccent@ineienhanced}{accent}
    \colorpreview{shibuigray@ineienhanced}{gray}
    \colorpreview{shibuidarkgray@ineienhanced}{darkgray}
    \colorpreview{shibuicodebg@ineienhanced}{codebg}
  \end{shibuiframe}

  \vspace{1em}

  Same deep sumi ink black as inei, with added visual effects: organic random light columns creating interplay of shadow and light, plus diagonal crosshatch texture for matte black paper finish.

  \vspace{0.5em}

  \textbf{Note:} Slower compilation due to TikZ effects. Use basic \texttt{inei} for faster builds.

  \vspace{0.5em}

  \textbf{Usage:} \texttt{\textbackslash usetheme[ineienhanced]\{Shibui\}}
\end{frame}

\begin{frame}{Color Scheme: ineilight}
  \begin{shibuiframe}{ineilight -- In'ei Light Variant}
    \colorpreview{shibuibg@ineilight}{bg}
    \colorpreview{shibuitext@ineilight}{text}
    \colorpreview{shibuiaccent@ineilight}{accent}
    \colorpreview{shibuigray@ineilight}{gray}
    \colorpreview{shibuidarkgray@ineilight}{darkgray}
    \colorpreview{shibuicodebg@ineilight}{codebg}
  \end{shibuiframe}

  \vspace{1em}

  Light variant with warm off-white background like aged Japanese paper. Maintains the muted aesthetic with same antique gold accent. Better for printing and readability.

  \vspace{0.5em}

  \textbf{Aesthetic:} Inverts the In'ei philosophy -- light as the canvas, shadows as the accent. Still contemplative and refined.

  \vspace{0.5em}

  \textbf{Usage:} \texttt{\textbackslash usetheme[ineilight]\{Shibui\}}
\end{frame}

%--------------------------------------------------
\section{Basic Content}
%--------------------------------------------------

\subsection{Text and Structure}

\begin{frame}{Lists \& Structure}
  \textbf{Itemized lists} use simple squares:

  \begin{itemize}
    \item First level items
    \begin{itemize}
      \item Second level items
      \begin{itemize}
        \item Third level items
      \end{itemize}
    \end{itemize}
  \end{itemize}

  \vspace{1em}

  \textbf{Enumerated lists} use simple numbering:

  \begin{enumerate}
    \item First item
    \item Second item
    \item Third item
  \end{enumerate}
\end{frame}

\begin{frame}{Text Formatting and Quotations}
  \begin{quote}
    ``Less is more.''
  \end{quote}

  \vspace{0.5em}
  \hfill --- Ludwig Mies van der Rohe

  \vspace{2em}

  \begin{quote}
    ``Simplicity is the ultimate sophistication.''
  \end{quote}

  \vspace{0.5em}
  \hfill --- Leonardo da Vinci
\end{frame}

\begin{frame}{Blocks \& Emphasis}
  \begin{block}{Standard Block}
    Blocks have minimal styling to avoid visual clutter.
  \end{block}

  \begin{exampleblock}{Example Block}
    Examples use slightly muted text for subtle differentiation.
  \end{exampleblock}

  \begin{alertblock}{Alert Block}
    Alerts use the accent color to draw attention when needed.
  \end{alertblock}

  \vspace{0.5em}

  Use blocks sparingly---content should speak for itself.
\end{frame}

\begin{frame}{Custom Shibui Boxes}
  The theme provides a custom \texttt{shibuiframe} environment with minimal styling:

  \vspace{0.5em}

  \begin{shibuiframe}{Key Concept}
    This is a custom box with only bottom and right borders, following Shibui's minimal aesthetic. The title overlaps the top-left corner with a clean border.

    \vspace{0.3em}

    Perfect for highlighting important concepts without visual clutter.
  \end{shibuiframe}

  \vspace{0.5em}

  \begin{shibuiframe}{Mathematical Insight}
    The box adapts to both light and dark themes automatically, maintaining readability and the clean Shibui style across all color schemes.
  \end{shibuiframe}
\end{frame}

\begin{frame}{Simple Tables}
  Tables follow the minimal Shibui aesthetic:

  \vspace{1em}

  \centering
  \begin{tabular}{lcc}
    \hline
    \textbf{Method} & \textbf{Accuracy} & \textbf{Time (ms)} \\
    \hline
    Algorithm A & 94.2\% & 12.3 \\
    Algorithm B & 96.7\% & 18.5 \\
    Algorithm C & 95.1\% & 15.2 \\
    \hline
  \end{tabular}

  \vspace{1em}

  \small Keep tables simple and uncluttered. Use horizontal lines sparingly.
\end{frame}

\begin{frame}{Advanced Tables}
  More complex data organization:

  \vspace{0.5em}

  \centering
  \begin{tabular}{llrr}
    \hline
    \textbf{Category} & \textbf{Item} & \textbf{Q1} & \textbf{Q2} \\
    \hline
    Revenue & Product A & \$125K & \$142K \\
            & Product B & \$89K  & \$95K \\
    \hline
    Costs   & Marketing & \$42K  & \$38K \\
            & Operations & \$67K & \$71K \\
    \hline
    \textbf{Net} & & \textbf{\$105K} & \textbf{\$128K} \\
    \hline
  \end{tabular}

  \vspace{1em}

  \small Align numbers to the right, text to the left.
\end{frame}

\begin{frame}{Two-Column Layout}
  The theme supports standard Beamer columns:

  \vspace{1em}

  \begin{columns}[T]
    \begin{column}{0.48\textwidth}
      \textbf{Left Column}

      \begin{itemize}
        \item Clean layout
        \item Easy to read
        \item Balanced design
      \end{itemize}
    \end{column}

    \begin{column}{0.48\textwidth}
      \textbf{Right Column}

      Use columns to:
      \begin{enumerate}
        \item Compare concepts
        \item Show before/after
        \item Present alternatives
      \end{enumerate}
    \end{column}
  \end{columns}
\end{frame}

\subsection{Mathematics}

\begin{frame}{Mathematics}
  The Shibui theme handles mathematical content elegantly:

  \vspace{1em}

  \textbf{Inline math}: The quadratic formula is $x = \frac{-b \pm \sqrt{b^2-4ac}}{2a}$.

  \vspace{1em}

  \textbf{Display equations}:
  \begin{equation}
    \int_{-\infty}^{\infty} e^{-x^2} \, dx = \sqrt{\pi}
  \end{equation}

  \vspace{0.5em}

  \textbf{Aligned equations}:
  \begin{align}
    \nabla \times \mathbf{E} &= -\frac{\partial \mathbf{B}}{\partial t} \\
    \nabla \times \mathbf{B} &= \mu_0 \mathbf{J} + \mu_0 \epsilon_0 \frac{\partial \mathbf{E}}{\partial t}
  \end{align}
\end{frame}

%--------------------------------------------------
\section{Academic \& Technical Content}
%--------------------------------------------------

\subsection{Academic Features}

\begin{frame}{Theorem Boxes}
  The theme provides styled theorem-like environments using theme colors:

  \vspace{0.5em}

  \begin{shibuitheorem}[Pythagorean Theorem]
    In a right triangle, the square of the hypotenuse equals the sum of squares of the other two sides:
    \[c^2 = a^2 + b^2\]
  \end{shibuitheorem}

  \vspace{0.3em}

  \begin{shibuilemma}
    If $n$ is even, then $n^2$ is even.
  \end{shibuilemma}

  \vspace{0.3em}

  All environments automatically use theme colors and adapt to the active color scheme.
\end{frame}

\begin{frame}{More Theorem Types}
  Additional theorem-like environments:

  \vspace{0.5em}

  \begin{shibuidefinition}[Limit]
    We say $\lim_{x \to a} f(x) = L$ if for every $\epsilon > 0$ there exists $\delta > 0$ such that $|f(x) - L| < \epsilon$ whenever $0 < |x - a| < \delta$.
  \end{shibuidefinition}

  \vspace{0.3em}

  \begin{shibuiproposition}
    The sum of two continuous functions is continuous.
  \end{shibuiproposition}

  \vspace{0.3em}

  \begin{shibuicorollary}
    All polynomial functions are continuous on $\mathbb{R}$.
  \end{shibuicorollary}
\end{frame}

\begin{frame}{Proof Environment}
  Proofs get a clean presentation with QED symbol:

  \vspace{0.5em}

  \begin{shibuitheorem}
    The sum of two even numbers is even.
  \end{shibuitheorem}

  \begin{shibuiproof}
    Let $m = 2k$ and $n = 2j$ be two even numbers where $k, j \in \mathbb{Z}$.

    Then $m + n = 2k + 2j = 2(k + j)$.

    Since $k + j \in \mathbb{Z}$, we have $m + n = 2(k + j)$ which is even by definition.
  \end{shibuiproof}

  \vspace{0.3em}

  The proof environment automatically adds proper spacing and the QED symbol.
\end{frame}

\begin{frame}[fragile, allowframebreaks]{Citations and Footnotes}
  The theme provides two citation styles:

  \vspace{0.5em}

  \textbf{Footnote citation (manual):} For academic presentations, the Shibui philosophy of minimalism was influenced by Japanese design principles.\footnote{Tanizaki, Jun'ichiro. \textit{In Praise of Shadows}. Leete's Island Books, 1977. \vspace{1em}}

  \vspace{0.5em}

  \begin{shibuiframe}{Usage}
    \textbf{Manual footnotes:} \\
    \texttt{\textbackslash footnote\{Author. \textbackslash textit\{Title\}. ...\}}

    \vspace{0.3em}

    \textbf{With bibliography (biblatex/natbib):} \\
    \texttt{\textbackslash footcite\{key\}} -- Full reference in footnote

    \vspace{0.3em}

    The \texttt{\textbackslash footcite} command works with biblatex, natbib, and standard bibtex, automatically adapting to your citation package.

    \vspace{0.3em}

    \small\textit{Note: To use \texttt{\textbackslash footcite}, load biblatex or natbib in the preamble and add your .bib file. Example:} \\
    \texttt{\textbackslash usepackage[backend=biber]\{biblatex\}} \\
    \texttt{\textbackslash addbibresource\{references.bib\}}
  \end{shibuiframe}

  \vspace{0.3em}

  \small Footnote citations keep slide content clean while providing full reference details.
\end{frame}

\begin{frame}[fragile, allowframebreaks]{Bibliography Setup}
  Setting up a bibliography with biblatex:

  \vspace{0.5em}

  \begin{shibuiframe}{Preamble Setup}
    \texttt{\textbackslash usepackage[backend=biber,style=authoryear]\{biblatex\}} \\
    \texttt{\textbackslash addbibresource\{references.bib\}}
  \end{shibuiframe}

  \vspace{0.5em}

  \begin{shibuiframe}{In Your Presentation}
    \texttt{\textbackslash footcite\{key\}} -- Footnote with full citation \\
    \texttt{\textbackslash cite\{key\}} -- Inline citation \\
    \texttt{\textbackslash parencite\{key\}} -- Citation in parentheses
  \end{shibuiframe}

  \vspace{0.5em}

  \begin{shibuiframe}{Print Bibliography}
    \texttt{\textbackslash begin\{frame\}\{References\}} \\
    \texttt{~~\textbackslash printbibliography} \\
    \texttt{\textbackslash end\{frame\}}
  \end{shibuiframe}

  \vspace{0.5em}

  \small Compile with: pdflatex → biber → pdflatex → pdflatex
\end{frame}

\begin{frame}{Academic Workflow Example}
  A complete academic presentation flow:

  \vspace{0.5em}

  \begin{enumerate}
    \item \textbf{State theorem}: Use \texttt{shibuitheorem} environment
    \item \textbf{Provide proof}: Use \texttt{shibuiproof} environment
    \item \textbf{Cite sources}: Use \texttt{\textbackslash footcite} for references
    \item \textbf{Add definitions}: Use \texttt{shibuidefinition} for terminology
    \item \textbf{Present corollaries}: Use \texttt{shibuicorollary} for consequences
  \end{enumerate}

  \vspace{1em}

  All environments work seamlessly together and adapt to your chosen color scheme, maintaining professional academic standards with minimalist Shibui aesthetics.
\end{frame}

\subsection{Code \& Algorithms}

\begin{frame}[fragile]{Code Listings}
  Clean, readable code with syntax highlighting:

  \vspace{0.5em}

  \begin{lstlisting}[language=Python]
# Fibonacci sequence generator
def fibonacci(n):
    """Generate first n Fibonacci numbers."""
    a, b = 0, 1
    for _ in range(n):
        yield a
        a, b = b, a + b

# Usage example
for num in fibonacci(10):
    print(num, end=' ')
  \end{lstlisting}
\end{frame}

\begin{frame}{Algorithms and Pseudocode}

  \begin{algorithm}[H]
    \small
    \caption{Binary Search}
    \begin{algorithmic}[1]
      \Procedure{BinarySearch}{$A, n, T$}
        \State $L \gets 0$
        \State $R \gets n - 1$
        \While{$L \leq R$}
          \State $m \gets \lfloor (L + R)/2 \rfloor$
          \If{$A[m] < T$}
            \State $L \gets m + 1$
          \ElsIf{$A[m] > T$}
            \State $R \gets m - 1$
          \Else
            \State \Return $m$
          \EndIf
        \EndWhile
        \State \Return \textit{unsuccessful}
      \EndProcedure
    \end{algorithmic}
  \end{algorithm}
\end{frame}

\subsection{Data Visualization}

\begin{frame}{Presenting Data}
  When including figures or data visualizations:

  \begin{itemize}
    \item Keep charts simple and uncluttered
    \item Remove unnecessary gridlines and decorations
    \item Use the theme's color palette for consistency
    \item Ensure high contrast for readability
    \item Add concise, informative captions
  \end{itemize}

  \vspace{1em}

  \textbf{Remember}: The goal is to illuminate, not decorate.
\end{frame}

\begin{frame}{Visual Hierarchy}
  Good presentations guide the eye naturally:

  \vspace{1em}

  \begin{enumerate}
    \item \textbf{Most important}: Use size and weight
    \item \emph{Moderate importance}: Use italics or color
    \item Least important: Use smaller text or gray
  \end{enumerate}

  \vspace{1em}

  The Shibui theme's generous whitespace helps create natural visual hierarchy without competing elements.
\end{frame}

\begin{frame}{TikZ Diagrams}
  \centering
  \begin{tikzpicture}[scale=0.8]
    % Nodes
    \node[circle,draw=shibuitext,fill=shibuicodebg,text=shibuitext,minimum size=1cm] (A) at (0,0) {A};
    \node[circle,draw=shibuitext,fill=shibuicodebg,text=shibuitext,minimum size=1cm] (B) at (3,1) {B};
    \node[circle,draw=shibuitext,fill=shibuicodebg,text=shibuitext,minimum size=1cm] (C) at (3,-1) {C};
    \node[circle,draw=shibuiaccent,fill=shibuicodebg,text=shibuitext,minimum size=1cm] (D) at (6,0) {D};

    % Edges
    \draw[->,thick,color=shibuitext] (A) -- (B) node[midway,above,text=shibuitext] {5};
    \draw[->,thick,color=shibuitext] (A) -- (C) node[midway,below,text=shibuitext] {3};
    \draw[->,thick,color=shibuitext] (B) -- (D) node[midway,above,text=shibuitext] {2};
    \draw[->,thick,color=shibuitext] (C) -- (D) node[midway,below,text=shibuitext] {4};
    \draw[->,thick,color=shibuiaccent] (B) -- (C) node[midway,right,text=shibuiaccent] {1};
  \end{tikzpicture}

  \vspace{0.5em}

  \small Weighted directed graph using themed colors (shibuitext, shibuiaccent, shibuicodebg)
\end{frame}

\begin{frame}{PGFPlots: Standard Style}
  \centering
  \begin{tikzpicture}
    \begin{axis}[
      width=0.75\textwidth,
      height=0.8\textheight,
      xlabel={$x$},
      ylabel={$f(x)$},
      grid=major,
      grid style={color=shibuigray!30},
      legend pos=north east,
      legend style={fill=shibuicodebg, draw=shibuitext, text=shibuitext},
      axis line style={color=shibuitext},
      tick label style={color=shibuitext},
      label style={color=shibuitext}
    ]
      \addplot[color=shibuitext, thick, smooth, domain=-2:2] {x^2};
      \addlegendentry{$x^2$}

      \addplot[color=shibuiaccent, thick, smooth, domain=-2:2] {2^x};
      \addlegendentry{$2^x$}

      \addplot[color=shibuidarkgray, thick, dashed, domain=-2:2] {0.5*x};
      \addlegendentry{$0.5x$}
    \end{axis}
  \end{tikzpicture}
\end{frame}

\begin{frame}{PGFPlots: Shibui Style}
  Simple activation with \texttt{shibui style} (no grid, legend outside):

  \vspace{0.3em}

  \centering
  \begin{tikzpicture}
    \begin{axis}[
      shibui style,
      shibui colors,
      width=0.75\textwidth,
      height=0.8\textheight,
      xlabel={$x$},
      ylabel={$f(x)$}
    ]
      \addplot+[smooth, domain=-2:2, samples=50] {x^2};
      \addlegendentry{$x^2$}

      \addplot+[smooth, domain=-2:2, samples=50] {2^x};
      \addlegendentry{$2^x$}

      \addplot+[smooth, domain=-2:2, samples=50] {0.5*x};
      \addlegendentry{$0.5x$}
    \end{axis}
  \end{tikzpicture}
\end{frame}

\begin{frame}{Dynamic Data Visualization}

  \vspace{0.3em}

  \centering
  \begin{tikzpicture}
    \begin{axis}[
      shibui style,
      width=0.75\textwidth,
      height=0.8\textheight,
      xlabel={Time ($t$)},
      ylabel={Signal Amplitude},
      ymin=-1.5, ymax=1.5
    ]
      % Simulated dynamic data
      \addplot[color=shibuitext, thick, smooth, domain=0:10, samples=100]
        {sin(deg(x))};
      \addlegendentry{Base signal}

      \addplot[color=shibuiaccent!70!shibuidarkgray, thick, dashed, smooth, domain=0:10, samples=100]
        {sin(deg(x)) + 0.2*sin(deg(5*x))};
      \addlegendentry{With noise}
    \end{axis}
  \end{tikzpicture}
\end{frame}

\begin{frame}{All Plot Colors Visualization}
  Showcasing all 5 theme plot colors with \texttt{shibui colors}:

  \vspace{0.3em}

  \centering
  \begin{tikzpicture}
    \begin{axis}[
      shibui style,
      shibui colors,
      width=0.75\textwidth,
      height=0.8\textheight,
      xlabel={$x$},
      ylabel={$f(x)$},
      ymin=-1.5, ymax=2,
      legend pos=north east
    ]
      % All 5 plot colors automatically applied
      \addplot+[smooth, domain=0:10, samples=100] {sin(deg(x))};
      \addlegendentry{shibuiplot1}

      \addplot+[smooth, domain=0:10, samples=100] {0.8*cos(deg(x)) + 0.5};
      \addlegendentry{shibuiplot2}

      \addplot+[smooth, domain=0:10, samples=100] {0.6*sin(deg(2*x)) - 0.5};
      \addlegendentry{shibuiplot3}

      \addplot+[smooth, domain=0:10, samples=100] {0.5*cos(deg(1.5*x)) + 1};
      \addlegendentry{shibuiplot4}

      \addplot+[smooth, domain=0:10, samples=100] {0.4*sin(deg(3*x)) - 1};
      \addlegendentry{shibuiplot5}
    \end{axis}
  \end{tikzpicture}
\end{frame}

%--------------------------------------------------
\section{Best Practices}
%--------------------------------------------------

\subsection{Design Principles}

\begin{frame}{Shibui Principles}
  Apply these principles when using this theme:

  \begin{itemize}
    \item \textbf{Show the content}: Let content take center stage
    \item \textbf{Reduce noise}: Every element should serve a purpose
    \item \textbf{Integrate text and graphics}: Create a unified whole
    \item \textbf{Respect your audience}: Clear, honest presentation
  \end{itemize}

  \vspace{1em}

  The theme provides the framework; you provide the clarity.
\end{frame}

\subsection{Technical Guidelines}

\begin{frame}{Technical Tips}
  \begin{itemize}
    \item Use \texttt{aspectratio=169} for widescreen (16:9)
    \item Use \texttt{aspectratio=43} for traditional (4:3)
    \item Uncomment \texttt{\textbackslash sectionframe} in theme for automatic section slides
    \item Keep slides focused---one main idea per slide
    \item Use generous spacing with \texttt{\textbackslash vspace}
  \end{itemize}
\end{frame}

%--------------------------------------------------
\section{Advanced Customization}
%--------------------------------------------------

\subsection{Callout Boxes}

\begin{frame}{Note Boxes}
  Use callout boxes to highlight important information:

  \vspace{0.5em}

  \begin{shibiunote}[Important]
    Callout boxes use theme colors and automatically adapt to your selected color scheme (nord, gruvbox, solarized, etc.).
  \end{shibiunote}

  \vspace{0.5em}

  \begin{shibiunote}[Remember]
    All theorem and callout environments are built with tcolorbox and support page breaks with the \texttt{breakable} option.
  \end{shibiunote}

  \vspace{0.5em}

  \textbf{Usage:} \texttt{\textbackslash begin\{shibiunote\}[Title] ... \textbackslash end\{shibiunote\}}
\end{frame}

\begin{frame}{Warning and Tip Boxes}
  Different callout types for different purposes:

  \vspace{0.5em}

  \begin{shibuiwarning}[Caution]
    Warning boxes use a more prominent style with slightly darker colors and thicker borders to draw attention.
  \end{shibuiwarning}

  \vspace{0.5em}

  \begin{shibuitip}[Pro Tip]
    Tip boxes use gray tones for subtle suggestions and helpful hints without being too distracting.
  \end{shibuitip}

  \vspace{0.5em}

  All callout boxes maintain the minimalist Shibui aesthetic while providing clear visual hierarchy.
\end{frame}

\subsection{Working with Theme Colors}

\begin{frame}[fragile]{Theme Color Reference}
  The theme defines semantic color names that adapt to your chosen color scheme:

  \vspace{0.5em}

  \begin{shibuiframe}{Available Theme Colors}
    \texttt{shibuibg} -- Main background color \\
    \texttt{shibuitext} -- Primary text color \\
    \texttt{shibuiaccent} -- Accent color for highlights \\
    \texttt{shibuigray} -- Light gray for subtle elements \\
    \texttt{shibuidarkgray} -- Dark gray for secondary text \\
    \texttt{shibuicodebg} -- Background for code blocks and boxes
  \end{shibuiframe}

  \vspace{0.5em}

  These colors automatically change when you switch color schemes (nord, solarized, gruvbox, etc.), ensuring consistent styling across your presentation.
\end{frame}

\begin{frame}[fragile]{Using Theme Colors in TikZ}
  Integrate theme colors into custom TikZ graphics:

  \vspace{0.5em}

  \begin{shibuiframe}{Example: Custom Diagram}
    \texttt{\textbackslash begin\{tikzpicture\}} \\
    \texttt{~~\textbackslash node[fill=shibuicodebg, draw=shibuiaccent,} \\
    \texttt{~~~~~~~~text=shibuitext] \{Content\};} \\
    \texttt{~~\textbackslash draw[color=shibuiaccent, thick] ...} \\
    \texttt{\textbackslash end\{tikzpicture\}}
  \end{shibuiframe}

  \vspace{0.5em}

  Using theme colors ensures your custom graphics match the presentation's color scheme automatically.
\end{frame}

\begin{frame}[fragile]{Using Theme Colors in PGFPlots}
  Apply theme colors to custom plots:

  \vspace{0.5em}

  \begin{shibuiframe}{Manual Color Assignment}
    \texttt{\textbackslash addplot[color=shibuiaccent, thick] \{...\};} \\
    \texttt{\textbackslash addplot[color=shibuidarkgray, dashed] \{...\};}
  \end{shibuiframe}

  \vspace{0.3em}

  \begin{shibuiframe}{Using Shibui Style}
    \texttt{\textbackslash begin\{axis\}[shibui style, shibui colors]} \\
    \texttt{~~\textbackslash addplot+ \{...\}; \% Uses cycle list} \\
    \texttt{\textbackslash end\{axis\}}
  \end{shibuiframe}

  \vspace{0.5em}

  The \texttt{shibui colors} cycle list provides theme-aware plot colors with varied line styles.
\end{frame}

\subsection{Custom Elements}

\begin{frame}[fragile]{Custom Footer Text and Logos}
  Personalize your presentation footer:

  \vspace{0.5em}

  \begin{shibuiframe}{Custom Footer}
    Add to preamble: \\
    \texttt{\textbackslash footertext\{Conference 2024 -- City, Country\}}
  \end{shibuiframe}

  \vspace{0.5em}

  \begin{shibuiframe}{Kanji Logo}
    Enable kanji logo: \\
    \texttt{\textbackslash usetheme[kanjilogo]\{Shibui\}}
  \end{shibuiframe}

  \vspace{0.5em}

  Footer text appears above the progress bar, aligned to the right. The kanji logo appears in the left margin if enabled.
\end{frame}

\begin{frame}[fragile]{Package Integration Tips}
  The Shibui theme works well with common Beamer packages:

  \vspace{0.5em}

  \begin{shibuiframe}{Recommended Packages}
    \textbf{Bibliography:} biblatex, natbib \\
    \textbf{Code:} listings (auto-styled), minted \\
    \textbf{Math:} amsmath, amsthm (compatible) \\
    \textbf{Graphics:} TikZ, PGFPlots (with shibui styles) \\
    \textbf{Tables:} booktabs (clean horizontal rules)
  \end{shibuiframe}

  \vspace{0.5em}

  \small The theme automatically configures listings and algorithms. For other packages, use theme colors for consistency.
\end{frame}

\subsection{Presenter Notes}

\begin{frame}{Notes Page Template}
  The theme includes a clean notes page template for dual-screen presentations:

  \vspace{0.5em}

  \begin{shibuiframe}{Using Presenter Notes}
    \textbf{Enable notes:} Add to preamble: \\
    \texttt{\textbackslash setbeameroption\{show notes on second screen\}}

    \vspace{0.3em}

    \textbf{Add notes to slides:} \\
    \texttt{\textbackslash note\{Your speaker notes here...\}}

    \vspace{0.3em}

    \textbf{Features:}
    \begin{itemize}
      \item Slide thumbnail preview
      \item Section and slide number in header
      \item Clean, readable layout
      \item Theme colors for consistency
    \end{itemize}
  \end{shibuiframe}
\end{frame}

\note{
  This is an example of speaker notes. They appear on the notes page with a preview of the current slide.

  You can use notes to:
  - Remind yourself of key points to mention
  - Include timing information
  - Add references or additional details
  - Prepare for potential questions
}

\subsection{Summary}

\begin{frame}{Summary}
  The Shibui Beamer theme offers:

  \begin{itemize}
    \item \textbf{Elegant design} inspired by Japanese minimalism
    \item \textbf{Minimal interface} that stays out of your way
    \item \textbf{Clear navigation} through sections and slides
    \item \textbf{Multiple font options} including Fira Sans, Garamond, Charter, and more
    \item \textbf{Calm color palettes} including Solarized variants
  \end{itemize}

  \vspace{1em}

  Use it to create presentations that respect both content and audience.
\end{frame}

\begin{frame}{Thank You}
  \centering
  \vspace{2em}

  {\Large Questions?}

  \vspace{2em}

  \textcolor{shibuidarkgray}{%
    This theme is available at: \\
    \texttt{github.com/yourname/shibui-beamer}
  }
\end{frame}

%--------------------------------------------------
% APPENDICES
%--------------------------------------------------

\appendix

\section{Appendix}

\begin{frame}[allowframebreaks]{Theme Options Reference}
  Complete list of theme options:

  \vspace{0.3em}

  \begin{shibuiframe}{Color Themes}
    \texttt{light} -- Light cream background (default) \\
    \texttt{dark} -- Dark background \\
    \texttt{solarizedlight}, \texttt{solarizeddark}, \texttt{solarizedosaka} -- Solarized variants \\
    \texttt{nord} -- Arctic blue theme \\
    \texttt{gruvboxlight}, \texttt{gruvboxdark} -- Warm retro colors \\
    \texttt{autumn} -- Warm earthy tones
  \end{shibuiframe}

  \vspace{0.3em}

  \begin{shibuiframe}{Font Themes}
    \texttt{sans} -- {\fontfamily{FiraSans-TLF}\selectfont Fira Sans} (default) \\
    \texttt{serif} -- {\fontfamily{ETbb-TLF}\selectfont ET Book}/{\fontfamily{ppl}\selectfont Palatino} \\
    \texttt{garamond} -- {\fontfamily{EBGaramond-TLF}\selectfont EB Garamond} \\
    \texttt{charter} -- {\fontfamily{bch}\selectfont Charter} \\
    \texttt{mono} -- {\fontfamily{FiraMono-TLF}\selectfont Fira Mono}
  \end{shibuiframe}

  \vspace{0.3em}

  \begin{shibuiframe}{Math Font Options}
    \texttt{mathdefault} -- Auto-match text font (default) \\
    \texttt{mathsans} -- Force sans-serif math (sfmath) \\
    \texttt{mathserif} -- Force serif math (newtxmath)
  \end{shibuiframe}

  \vspace{0.3em}

  \begin{shibuiframe}{Layout Options}
    \texttt{nosectionheader} -- Hide section name from header \\
    \texttt{navontop} -- Show navigation squares above title \\
    \texttt{nosub} -- Hide subsections in table of contents
  \end{shibuiframe}

  \vspace{0.3em}

  \begin{shibuiframe}{Progress Bar Options}
    \texttt{progressbar=basic} -- Single continuous bar (default) \\
    \texttt{progressbar=segmented} -- Section-based segments \\
    \texttt{progressbar=none} -- No progress bar, page numbers only
  \end{shibuiframe}

  \vspace{0.3em}

  \textbf{Usage:} \texttt{\textbackslash usetheme[nord,sans,navontop]\{Shibui\}} \\
  \textbf{Example:} \texttt{\textbackslash usetheme[progressbar=segmented]\{Shibui\}}
\end{frame}

\begin{frame}{Color Definitions}
  Theme colors available for use:

  \vspace{1em}

  \centering
  \begin{tabular}{ll}
    \hline
    \textbf{Color Name} & \textbf{Usage} \\
    \hline
    \texttt{shibuibg} & Background color \\
    \texttt{shibuitext} & Main text color \\
    \texttt{shibuiaccent} & Accent/highlight color \\
    \texttt{shibuigray} & UI elements (light) \\
    \texttt{shibuidarkgray} & Secondary text \\
    \texttt{shibuicodebg} & Code/box background \\
    \hline
  \end{tabular}

  \vspace{1em}

  \small Use with: \texttt{\textbackslash color\{shibuiaccent\}} or \texttt{\textbackslash textcolor\{shibuiaccent\}\{text\}}
\end{frame}

\begin{frame}[allowframebreaks]{Custom Environments}
  Special environments provided by the theme:

  \vspace{0.5em}

  \begin{shibuiframe}{shibuiframe}
    Minimal box with background color for highlighting concepts.

    \textbf{Usage:} \texttt{\textbackslash begin\{shibuiframe\}\{Title\} ... \textbackslash end\{shibuiframe\}}
  \end{shibuiframe}

  \vspace{0.3em}

  \begin{shibuiframe}{Theorem Environments}
    \texttt{shibuitheorem[title]} -- Theorem box \\
    \texttt{shibuilemma[title]} -- Lemma box \\
    \texttt{shibuicorollary[title]} -- Corollary box \\
    \texttt{shibuidefinition[title]} -- Definition box \\
    \texttt{shibuiproposition[title]} -- Proposition box \\
    \texttt{shibuiproof} -- Proof with QED symbol
  \end{shibuiframe}

  \vspace{0.3em}

  \begin{shibuiframe}{Callout Boxes}
    \texttt{shibiunote[title]} -- Note/info box \\
    \texttt{shibuiwarning[title]} -- Warning box \\
    \texttt{shibuitip[title]} -- Tip/hint box
  \end{shibuiframe}

  \vspace{0.3em}

  \begin{shibuiframe}{Citation Commands}
    \texttt{\textbackslash cite\{key\}} -- Normal inline citation \\
    \texttt{\textbackslash footcite\{key\}} -- Footnote citation with full reference
  \end{shibuiframe}

  \vspace{0.5em}

  All environments use theme colors and adapt to the active color scheme. Standard Beamer environments also work: blocks, columns, overlays, etc.
\end{frame}

\begin{frame}{Custom Footer Text}
  Add custom text or logos above the progress bar:

  \vspace{0.5em}

  \begin{shibuiframe}{footertext Command}
    The \texttt{\textbackslash footertext\{\}} command lets you add custom text displayed above the progress bar in very small font.

    \vspace{0.3em}

    \textbf{Usage in preamble:}

    \texttt{\textbackslash footertext\{Conference 2024 -- Paris\}}

    \vspace{0.3em}

    Perfect for adding: conference names, dates, institutional logos, or copyright notices.
  \end{shibuiframe}

  \vspace{0.3em}

  The footer text appears on all slides except the title page and uses the secondary text color for minimal visual impact.
\end{frame}

\begin{frame}{Additional Resources}
  \begin{itemize}
    \item \textbf{Design References}
      \begin{itemize}
        \item Japanese aesthetics: Wabi-sabi, Shibui, Ma
        \item Minimalist design principles
        \item Typography in presentation design
      \end{itemize}

    \vspace{0.5em}

    \item \textbf{Color Schemes}
      \begin{itemize}
        \item Solarized color palette
        \item Nord theme
        \item Gruvbox retro colors
      \end{itemize}

    \vspace{0.5em}

    \item \textbf{Beamer Documentation}
      \begin{itemize}
        \item Official Beamer user guide
        \item Theme development guide
      \end{itemize}
  \end{itemize}
\end{frame}

\begin{frame}[allowframebreaks]{Troubleshooting}
  \begin{shibuiframe}{Font Issues}
    If ET Book font is not available, Palatino is used as fallback. For Fira Sans, ensure the package is installed: \texttt{texlive-fonts-extra}
  \end{shibuiframe}

  \vspace{0.5em}

  \begin{shibuiframe}{Color Problems}
    Ensure you're using XeLaTeX or pdfLaTeX with appropriate color packages. Theme loads colors automatically.
  \end{shibuiframe}

  \vspace{0.5em}

  \begin{shibuiframe}{Progress Bar}
    Progress bar is automatically hidden on title page and plain frames. Three modes available:

    \begin{itemize}
      \item \textbf{basic} (default): Single continuous bar
      \item \textbf{segmented}: Section-based segments with gaps. Past sections are dimmed, current section fills progressively
      \item \textbf{none}: No bar, only page numbers
    \end{itemize}

    Use \texttt{noframenumbering} option to exclude specific frames from count.
  \end{shibuiframe}
\end{frame}

\end{document}
