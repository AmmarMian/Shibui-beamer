\documentclass[aspectratio=169]{beamer}

% Use the Tufte theme
% Options:
%   - Color: dark (default) or light
%   - Font: serif (default) or sans
\usetheme[light, sans, nosub]{Tufte}

% Additional packages for examples
\usepackage{listings}
\usepackage{algorithm}
\usepackage{algpseudocode}
\usepackage{pgfplots}
\pgfplotsset{compat=1.18}

% NOTE: Listings and algorithm styling are now handled automatically by the theme!
% You can still override settings here if needed.

% Metadata
\title{The Tufte Beamer Theme}
\subtitle{A Minimalist Design for Clear Communication}
\author{Your Name}
\institute{Your Institution}
\date{\today}

\begin{document}

% Title page (automatically formatted, no header/footer)
\maketitle

% Table of contents
\begin{frame}{Overview}
  \tableofcontents
\end{frame}

%--------------------------------------------------
\section{Introduction}
%--------------------------------------------------

% Optional: uncomment to show section slide
% \sectionframe

\begin{frame}{Design Philosophy}
  The Tufte theme embodies Edward Tufte's design principles:

  \begin{itemize}
    \item Clarity through simplicity
    \item Generous use of whitespace
    \item Elegant typography
    \item Minimal but effective navigation
    \item Focus on content, not decoration
  \end{itemize}

  \vspace{1em}
  This theme is inspired by the \textcolor{tufteaccent}{tufte-css} project and uses a clean, readable color palette.
\end{frame}

\begin{frame}{Key Features}
  \begin{enumerate}
    \item \textbf{Dark Theme Colors}: Dark background with light, readable text
    \item \textbf{Navigation Squares}: Shows current section slides only
    \item \textbf{Progress Bar}: Visual indicator with page numbers
    \item \textbf{Minimal Headline}: Section and title without visual clutter
    \item \textbf{ET Book Font}: Beautiful serif typography (Palatino fallback)
  \end{enumerate}
\end{frame}

%--------------------------------------------------
\section{Typography \& Color}
%--------------------------------------------------

\subsection{Typography}

\begin{frame}{Typography Matters}
  Good typography is essential for effective presentations.

  \vspace{1em}

  \textbf{This theme uses serif fonts} in the tradition of Tufte's books:

  \begin{itemize}
    \item Primary: ET Book (if available)
    \item Fallback: Palatino
    \item Both provide excellent readability
  \end{itemize}

  \vspace{1em}

  \emph{Emphasis through italics} rather than bold helps maintain visual calm.
\end{frame}

\subsection{Color Schemes}

\begin{frame}{Color Palette}
  The theme supports both \textbf{light} and \textbf{dark} color schemes:

  \vspace{0.5em}

  \begin{block}{Dark Theme (Current)}
    \textbf{Background}: Dark (\texttt{\#1A1A1A}) \\
    \textbf{Text}: Light (\texttt{\#EEEEEE}) \\
    \textbf{Accent}: Lighter red (\texttt{\#FF6B6B})
  \end{block}

  \vspace{0.5em}

  \begin{exampleblock}{Light Theme}
    \textbf{Background}: Warm cream (\texttt{\#FFFFF8}) \\
    \textbf{Text}: Near black (\texttt{\#111111}) \\
    \textbf{Accent}: Dark red (\texttt{\#A00000})
  \end{exampleblock}

  \vspace{0.5em}

  Both palettes provide excellent contrast and reduce eye strain.
\end{frame}

%--------------------------------------------------
\section{Content Examples}
%--------------------------------------------------

\subsection{Lists and Structure}

\begin{frame}{Lists \& Structure}
  \textbf{Itemized lists} use simple squares:

  \begin{itemize}
    \item First level items
    \begin{itemize}
      \item Second level items
      \begin{itemize}
        \item Third level items
      \end{itemize}
    \end{itemize}
  \end{itemize}

  \vspace{1em}

  \textbf{Enumerated lists} use simple numbering:

  \begin{enumerate}
    \item First item
    \item Second item
    \item Third item
  \end{enumerate}
\end{frame}

\subsection{Blocks and Boxes}

\begin{frame}{Blocks \& Emphasis}
  \begin{block}{Standard Block}
    Blocks have minimal styling to avoid visual clutter.
  \end{block}

  \begin{exampleblock}{Example Block}
    Examples use slightly muted text for subtle differentiation.
  \end{exampleblock}

  \begin{alertblock}{Alert Block}
    Alerts use the accent color to draw attention when needed.
  \end{alertblock}

  \vspace{0.5em}

  Use blocks sparingly---content should speak for itself.
\end{frame}

\begin{frame}{Custom Tufte Boxes}
  The theme provides a custom \texttt{tufteframe} environment with minimal styling:

  \vspace{0.5em}

  \begin{tufteframe}{Key Concept}
    This is a custom box with only bottom and right borders, following Tufte's minimal aesthetic. The title overlaps the top-left corner with a clean border.

    \vspace{0.3em}

    Perfect for highlighting important concepts without visual clutter.
  \end{tufteframe}

  \vspace{0.5em}

  \begin{tufteframe}{Mathematical Insight}
    The box adapts to both light and dark themes automatically, maintaining readability and the clean Tufte style across all color schemes.
  \end{tufteframe}
\end{frame}

\subsection{Tables}

\begin{frame}{Simple Tables}
  Tables follow the minimal Tufte aesthetic:

  \vspace{1em}

  \centering
  \begin{tabular}{lcc}
    \hline
    \textbf{Method} & \textbf{Accuracy} & \textbf{Time (ms)} \\
    \hline
    Algorithm A & 94.2\% & 12.3 \\
    Algorithm B & 96.7\% & 18.5 \\
    Algorithm C & 95.1\% & 15.2 \\
    \hline
  \end{tabular}

  \vspace{1em}

  \small Keep tables simple and uncluttered. Use horizontal lines sparingly.
\end{frame}

\begin{frame}{Advanced Tables}
  More complex data organization:

  \vspace{0.5em}

  \centering
  \begin{tabular}{llrr}
    \hline
    \textbf{Category} & \textbf{Item} & \textbf{Q1} & \textbf{Q2} \\
    \hline
    Revenue & Product A & \$125K & \$142K \\
            & Product B & \$89K  & \$95K \\
    \hline
    Costs   & Marketing & \$42K  & \$38K \\
            & Operations & \$67K & \$71K \\
    \hline
    \textbf{Net} & & \textbf{\$105K} & \textbf{\$128K} \\
    \hline
  \end{tabular}

  \vspace{1em}

  \small Align numbers to the right, text to the left.
\end{frame}

\begin{frame}{Two-Column Layout}
  The theme supports standard Beamer columns:

  \vspace{1em}

  \begin{columns}[T]
    \begin{column}{0.48\textwidth}
      \textbf{Left Column}

      \begin{itemize}
        \item Clean layout
        \item Easy to read
        \item Balanced design
      \end{itemize}
    \end{column}

    \begin{column}{0.48\textwidth}
      \textbf{Right Column}

      Use columns to:
      \begin{enumerate}
        \item Compare concepts
        \item Show before/after
        \item Present alternatives
      \end{enumerate}
    \end{column}
  \end{columns}
\end{frame}

\subsection{Text Elements}

\begin{frame}{Quotations}
  \begin{quote}
    ``The commonality between science and art is in trying to see profoundly---to develop strategies of seeing and showing.''
  \end{quote}

  \vspace{0.5em}
  \hfill --- Edward Tufte

  \vspace{2em}

  \begin{quote}
    ``Above all else show the data.''
  \end{quote}

  \vspace{0.5em}
  \hfill --- Edward Tufte
\end{frame}

\subsection{Mathematics}

\begin{frame}{Mathematics}
  The Tufte theme handles mathematical content elegantly:

  \vspace{1em}

  \textbf{Inline math}: The quadratic formula is $x = \frac{-b \pm \sqrt{b^2-4ac}}{2a}$.

  \vspace{1em}

  \textbf{Display equations}:
  \begin{equation}
    \int_{-\infty}^{\infty} e^{-x^2} \, dx = \sqrt{\pi}
  \end{equation}

  \vspace{0.5em}

  \textbf{Aligned equations}:
  \begin{align}
    \nabla \times \mathbf{E} &= -\frac{\partial \mathbf{B}}{\partial t} \\
    \nabla \times \mathbf{B} &= \mu_0 \mathbf{J} + \mu_0 \epsilon_0 \frac{\partial \mathbf{E}}{\partial t}
  \end{align}
\end{frame}

\subsection{Code and Algorithms}

\begin{frame}[fragile]{Code Listings}
  Clean, readable code with syntax highlighting:

  \vspace{0.5em}

  \begin{lstlisting}[language=Python]
# Fibonacci sequence generator
def fibonacci(n):
    """Generate first n Fibonacci numbers."""
    a, b = 0, 1
    for _ in range(n):
        yield a
        a, b = b, a + b

# Usage example
for num in fibonacci(10):
    print(num, end=' ')
  \end{lstlisting}
\end{frame}

%--------------------------------------------------
\section{Data \& Figures}
%--------------------------------------------------

\subsection{Data Visualization Principles}

\begin{frame}{Presenting Data}
  When including figures or data visualizations:

  \begin{itemize}
    \item Keep charts simple and uncluttered
    \item Remove unnecessary gridlines and decorations
    \item Use the theme's color palette for consistency
    \item Ensure high contrast for readability
    \item Add concise, informative captions
  \end{itemize}

  \vspace{1em}

  \textbf{Remember}: The goal is to illuminate, not decorate.
\end{frame}

\begin{frame}{Visual Hierarchy}
  Good presentations guide the eye naturally:

  \vspace{1em}

  \begin{enumerate}
    \item \textbf{Most important}: Use size and weight
    \item \emph{Moderate importance}: Use italics or color
    \item Least important: Use smaller text or gray
  \end{enumerate}

  \vspace{1em}

  The Tufte theme's generous whitespace helps create natural visual hierarchy without competing elements.
\end{frame}

\subsection{Algorithms and Pseudocode}

\begin{frame}{Algorithms}
  \begin{algorithm}[H]
    \caption{Binary Search}
    \begin{algorithmic}[1]
      \Procedure{BinarySearch}{$A, n, T$}
        \State $L \gets 0$
        \State $R \gets n - 1$
        \While{$L \leq R$}
          \State $m \gets \lfloor (L + R)/2 \rfloor$
          \If{$A[m] < T$}
            \State $L \gets m + 1$
          \ElsIf{$A[m] > T$}
            \State $R \gets m - 1$
          \Else
            \State \Return $m$
          \EndIf
        \EndWhile
        \State \Return \textit{unsuccessful}
      \EndProcedure
    \end{algorithmic}
  \end{algorithm}
\end{frame}

\subsection{Graphics and Plots}

\begin{frame}{TikZ Diagrams}
  \centering
  \begin{tikzpicture}[scale=0.8]
    % Nodes
    \node[circle,draw,fill=tuftegray!20,minimum size=1cm] (A) at (0,0) {A};
    \node[circle,draw,fill=tuftegray!20,minimum size=1cm] (B) at (3,1) {B};
    \node[circle,draw,fill=tuftegray!20,minimum size=1cm] (C) at (3,-1) {C};
    \node[circle,draw,fill=tuftegray!20,minimum size=1cm] (D) at (6,0) {D};

    % Edges
    \draw[->,thick,color=tuftetext] (A) -- (B) node[midway,above] {5};
    \draw[->,thick,color=tuftetext] (A) -- (C) node[midway,below] {3};
    \draw[->,thick,color=tuftetext] (B) -- (D) node[midway,above] {2};
    \draw[->,thick,color=tuftetext] (C) -- (D) node[midway,below] {4};
    \draw[->,thick,color=tufteaccent] (B) -- (C) node[midway,right] {1};
  \end{tikzpicture}

  \vspace{0.5em}

  \small Weighted directed graph showing minimal Tufte-inspired styling
\end{frame}

\begin{frame}{PGFPlots: Standard Style}
  \centering
  \begin{tikzpicture}
    \begin{axis}[
      width=0.75\textwidth,
      height=0.55\textheight,
      xlabel={$x$},
      ylabel={$f(x)$},
      grid=major,
      grid style={color=tuftegray!30},
      legend pos=north west,
      axis line style={color=tuftetext},
      tick label style={color=tuftetext},
      label style={color=tuftetext}
    ]
      \addplot[color=tuftetext, thick, smooth, domain=-2:2] {x^2};
      \addlegendentry{$x^2$}

      \addplot[color=tufteaccent, thick, smooth, domain=-2:2] {2^x};
      \addlegendentry{$2^x$}

      \addplot[color=tuftedarkgray, thick, dashed, domain=-2:2] {0.5*x};
      \addlegendentry{$0.5x$}
    \end{axis}
  \end{tikzpicture}
\end{frame}

\begin{frame}{PGFPlots: Tufte Style}
  Simple activation with \texttt{tufte style} keyword:

  \vspace{0.3em}

  \centering
  \begin{tikzpicture}
    \begin{axis}[
      tufte style,
      tufte colors,
      width=0.75\textwidth,
      height=0.5\textheight,
      xlabel={$x$},
      ylabel={$f(x)$},
      legend pos=north west
    ]
      \addplot[smooth, domain=-2:2, samples=50] {x^2};
      \addlegendentry{$x^2$}

      \addplot[smooth, domain=-2:2, samples=50] {2^x};
      \addlegendentry{$2^x$}

      \addplot[smooth, domain=-2:2, samples=50] {0.5*x};
      \addlegendentry{$0.5x$}
    \end{axis}
  \end{tikzpicture}
\end{frame}

\begin{frame}{Dynamic Data Visualization}
  Tufte-style plot with dynamically sized data:

  \vspace{0.3em}

  \centering
  \begin{tikzpicture}
    \begin{axis}[
      tufte style,
      tufte grid,
      width=0.75\textwidth,
      height=0.52\textheight,
      xlabel={Time ($t$)},
      ylabel={Signal Amplitude},
      legend pos=south east,
      ymin=-1.5, ymax=1.5
    ]
      % Simulated dynamic data
      \addplot[color=tuftetext, thick, smooth, domain=0:10, samples=100]
        {sin(deg(x))};
      \addlegendentry{Base signal}

      \addplot[color=tufteaccent, thick, smooth, domain=0:10, samples=100]
        {sin(deg(x)) + 0.2*sin(deg(5*x))};
      \addlegendentry{With noise}
    \end{axis}
  \end{tikzpicture}
\end{frame}

%--------------------------------------------------
\section{Best Practices}
%--------------------------------------------------

\subsection{Design Principles}

\begin{frame}{Tufte's Principles}
  Apply these principles when using this theme:

  \begin{itemize}
    \item \textbf{Show the data}: Let content take center stage
    \item \textbf{Reduce noise}: Every element should serve a purpose
    \item \textbf{Integrate text and graphics}: Create a unified whole
    \item \textbf{Respect your audience}: Clear, honest presentation
  \end{itemize}

  \vspace{1em}

  The theme provides the framework; you provide the clarity.
\end{frame}

\subsection{Technical Guidelines}

\begin{frame}{Technical Tips}
  \begin{itemize}
    \item Use \texttt{aspectratio=169} for widescreen (16:9)
    \item Use \texttt{aspectratio=43} for traditional (4:3)
    \item Uncomment \texttt{\textbackslash sectionframe} in theme for automatic section slides
    \item Keep slides focused---one main idea per slide
    \item Use generous spacing with \texttt{\textbackslash vspace}
  \end{itemize}
\end{frame}

%--------------------------------------------------
\section{Conclusion}
%--------------------------------------------------

\begin{frame}{Summary}
  The Tufte Beamer theme offers:

  \begin{itemize}
    \item \textbf{Elegant design} inspired by Tufte's work
    \item \textbf{Minimal interface} that stays out of your way
    \item \textbf{Clear navigation} through sections and slides
    \item \textbf{Readable typography} with ET Book/Palatino
    \item \textbf{Calm color palette} from tufte-css
  \end{itemize}

  \vspace{1em}

  Use it to create presentations that respect both content and audience.
\end{frame}

\begin{frame}{Thank You}
  \centering
  \vspace{2em}

  {\Large Questions?}

  \vspace{2em}

  \textcolor{tuftedarkgray}{%
    This theme is available at: \\
    \texttt{github.com/yourname/tufte-beamer}
  }
\end{frame}

%--------------------------------------------------
% APPENDICES
%--------------------------------------------------

\appendix

\section{Appendix}

\begin{frame}{Theme Options Reference}
  Complete list of theme options:

  \vspace{0.5em}

  \begin{tufteframe}{Color Themes}
    \texttt{dark} -- Dark background (default) \\
    \texttt{light} -- Light cream background
  \end{tufteframe}

  \vspace{0.5em}

  \begin{tufteframe}{Font Themes}
    \texttt{serif} -- ET Book/Palatino (default) \\
    \texttt{sans} -- Fira Sans
  \end{tufteframe}

  \vspace{0.5em}

  \textbf{Usage:} \texttt{\textbackslash usetheme[light,sans]\{Tufte\}}
\end{frame}

\begin{frame}{Color Definitions}
  Theme colors available for use:

  \vspace{1em}

  \centering
  \begin{tabular}{ll}
    \hline
    \textbf{Color Name} & \textbf{Usage} \\
    \hline
    \texttt{tuftebg} & Background color \\
    \texttt{tuftetext} & Main text color \\
    \texttt{tufteaccent} & Accent/highlight color \\
    \texttt{tuftegray} & UI elements (light) \\
    \texttt{tuftedarkgray} & Secondary text \\
    \texttt{tuftecodebg} & Code/box background \\
    \hline
  \end{tabular}

  \vspace{1em}

  \small Use with: \texttt{\textbackslash color\{tufteaccent\}} or \texttt{\textbackslash textcolor\{tufteaccent\}\{text\}}
\end{frame}

\begin{frame}{Custom Environments}
  Special environments provided by the theme:

  \vspace{0.5em}

  \begin{tufteframe}{tufteframe}
    Minimal box with background color for highlighting concepts.

    \textbf{Usage:}

    \texttt{\textbackslash begin\{tufteframe\}\{Title\}} \\
    \texttt{~~Content...} \\
    \texttt{\textbackslash end\{tufteframe\}}
  \end{tufteframe}

  \vspace{0.5em}

  All standard Beamer environments work as expected: blocks, columns, overlays, etc.
\end{frame}

\begin{frame}{Additional Resources}
  \begin{itemize}
    \item \textbf{Edward Tufte's Books}
      \begin{itemize}
        \item \emph{The Visual Display of Quantitative Information}
        \item \emph{Envisioning Information}
        \item \emph{Beautiful Evidence}
      \end{itemize}

    \vspace{0.5em}

    \item \textbf{Related Projects}
      \begin{itemize}
        \item tufte-latex: Document classes for LaTeX
        \item tufte-css: Web design framework
      \end{itemize}

    \vspace{0.5em}

    \item \textbf{Beamer Documentation}
      \begin{itemize}
        \item Official Beamer user guide
        \item Theme development guide
      \end{itemize}
  \end{itemize}
\end{frame}

\begin{frame}{Troubleshooting}
  \begin{tufteframe}{Font Issues}
    If ET Book font is not available, Palatino is used as fallback. For Fira Sans, ensure the package is installed: \texttt{texlive-fonts-extra}
  \end{tufteframe}

  \vspace{0.5em}

  \begin{tufteframe}{Color Problems}
    Ensure you're using XeLaTeX or pdfLaTeX with appropriate color packages. Theme loads colors automatically.
  \end{tufteframe}

  \vspace{0.5em}

  \begin{tufteframe}{Progress Bar}
    Progress bar is automatically hidden on title page and plain frames. Use \texttt{noframenumbering} option to exclude specific frames.
  \end{tufteframe}
\end{frame}

\end{document}
