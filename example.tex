\documentclass[aspectratio=169]{beamer}

% Use the Shibui theme - Japanese aesthetic of simple, subtle beauty
% Options:
%   - Color: light (default), dark, solarizedlight, solarizeddark, solarizedosaka,
%            nord, gruvboxlight, gruvboxdark, autumn
%   - Font: sans (default), serif, garamond, charter, mono
%   - Layout: nosectionheader (default, hide section from header), navontop (show navigation above title)
%   - TOC: nosub (hide subsections in table of contents)
%   - Progress bar: progressbar=basic (default), progressbar=segmented, progressbar=none
%
% Examples:
%   \usetheme{Shibui}                                   % Use all defaults
%   \usetheme[solarizedosaka]{Shibui}                   % Just color scheme
%   \usetheme[nord, serif]{Shibui}                      % Color + font
%   \usetheme[dark, navontop]{Shibui}                   % Show navigation squares above title
%   \usetheme[gruvboxdark, garamond]{Shibui}            % Multiple options
%   \usetheme[nord, progressbar=segmented]{Shibui}      % Section-based progress bar
\usetheme[sans, nosub, progressbar=segmented, kanjilogo]{Shibui}

% Custom footer text (optional)
\footertext{Conference Name 2024 -- City, Country}

% Additional packages for examples
\usepackage{listings}
\usepackage{algorithm}
\usepackage{algpseudocode}
\usepackage{pgfplots}
\pgfplotsset{compat=1.18}

% NOTE: Listings and algorithm styling are now handled automatically by the theme!
% You can still override settings here if needed.

% Metadata
\title{The Shibui Beamer Theme}
\subtitle{A Minimalist Design for Clear Communication}
\author{Your Name}
\institute{Your Institution}
\date{\today}

\begin{document}

% Title page (automatically formatted, no header/footer)
\maketitle

% Table of contents
\begin{frame}{Overview}
  \tableofcontents
\end{frame}

%--------------------------------------------------
\section{Introduction}
%--------------------------------------------------

% Optional: uncomment to show section slide
% \sectionframe

\begin{frame}{Design Philosophy}
  The Shibui theme embodies Japanese minimalist design principles:

  \begin{itemize}
    \item Clarity through simplicity
    \item Generous use of whitespace
    \item Elegant typography
    \item Minimal but effective navigation
    \item Focus on content, not decoration
  \end{itemize}

  \vspace{1em}
  Shibui represents simple, subtle, and unobtrusive beauty through refined restraint.
\end{frame}

\begin{frame}{Key Features}
  \begin{enumerate}
    \item \textbf{Dark Theme Colors}: Dark background with light, readable text
    \item \textbf{Navigation Squares}: Shows current section slides only
    \item \textbf{Progress Bar}: Visual indicator with page numbers
    \item \textbf{Minimal Headline}: Section and title without visual clutter
    \item \textbf{Multiple Fonts}: Choose from sans-serif, serif, or monospace options
  \end{enumerate}
\end{frame}

%--------------------------------------------------
\section{Typography \& Color}
%--------------------------------------------------

\subsection{Typography}

\begin{frame}{Typography Matters}
  Good typography is essential for effective presentations.

  \vspace{1em}

  \textbf{This theme offers 5 font options} for different presentation styles:

  \begin{itemize}
    \item \textbf{sans} (default): Fira Sans -- clean, modern sans-serif
    \item \textbf{serif}: ET Book or Palatino fallback -- classic elegance
    \item \textbf{garamond}: EB Garamond -- sophisticated serif typography
    \item \textbf{charter}: Charter -- highly readable serif
    \item \textbf{mono}: Fira Mono -- monospace for code-heavy presentations
  \end{itemize}

  \vspace{1em}

  \emph{Emphasis through italics} rather than bold helps maintain visual calm.
\end{frame}

\subsection{Color Schemes}

\begin{frame}{Color Palette}
  The theme now supports \textbf{9 color schemes} for different aesthetics:

  \vspace{0.5em}

  \begin{block}{Original Themes}
    \textbf{dark} (default): Dark background with light text and red accent \\
    \textbf{light}: Warm cream background with near-black text and dark red accent
  \end{block}

  \vspace{0.5em}

  \begin{exampleblock}{New Color Schemes}
    \textbf{solarizedlight}, \textbf{solarizeddark}, \textbf{solarizedosaka}: Solarized variants \\
    \textbf{nord}: Arctic, north-bluish color palette \\
    \textbf{gruvboxlight}, \textbf{gruvboxdark}: Retro groove with warm tones \\
    \textbf{autumn}: Warm autumn colors with browns and oranges
  \end{exampleblock}

  \vspace{0.3em}

  All palettes provide excellent contrast and reduce eye strain.
\end{frame}

%--------------------------------------------------
\section{Content Examples}
%--------------------------------------------------

\subsection{Lists and Structure}

\begin{frame}{Lists \& Structure}
  \textbf{Itemized lists} use simple squares:

  \begin{itemize}
    \item First level items
    \begin{itemize}
      \item Second level items
      \begin{itemize}
        \item Third level items
      \end{itemize}
    \end{itemize}
  \end{itemize}

  \vspace{1em}

  \textbf{Enumerated lists} use simple numbering:

  \begin{enumerate}
    \item First item
    \item Second item
    \item Third item
  \end{enumerate}
\end{frame}

\subsection{Blocks and Boxes}

\begin{frame}{Blocks \& Emphasis}
  \begin{block}{Standard Block}
    Blocks have minimal styling to avoid visual clutter.
  \end{block}

  \begin{exampleblock}{Example Block}
    Examples use slightly muted text for subtle differentiation.
  \end{exampleblock}

  \begin{alertblock}{Alert Block}
    Alerts use the accent color to draw attention when needed.
  \end{alertblock}

  \vspace{0.5em}

  Use blocks sparingly---content should speak for itself.
\end{frame}

\begin{frame}{Custom Shibui Boxes}
  The theme provides a custom \texttt{shibuiframe} environment with minimal styling:

  \vspace{0.5em}

  \begin{shibuiframe}{Key Concept}
    This is a custom box with only bottom and right borders, following Shibui's minimal aesthetic. The title overlaps the top-left corner with a clean border.

    \vspace{0.3em}

    Perfect for highlighting important concepts without visual clutter.
  \end{shibuiframe}

  \vspace{0.5em}

  \begin{shibuiframe}{Mathematical Insight}
    The box adapts to both light and dark themes automatically, maintaining readability and the clean Shibui style across all color schemes.
  \end{shibuiframe}
\end{frame}

\subsection{Tables}

\begin{frame}{Simple Tables}
  Tables follow the minimal Shibui aesthetic:

  \vspace{1em}

  \centering
  \begin{tabular}{lcc}
    \hline
    \textbf{Method} & \textbf{Accuracy} & \textbf{Time (ms)} \\
    \hline
    Algorithm A & 94.2\% & 12.3 \\
    Algorithm B & 96.7\% & 18.5 \\
    Algorithm C & 95.1\% & 15.2 \\
    \hline
  \end{tabular}

  \vspace{1em}

  \small Keep tables simple and uncluttered. Use horizontal lines sparingly.
\end{frame}

\begin{frame}{Advanced Tables}
  More complex data organization:

  \vspace{0.5em}

  \centering
  \begin{tabular}{llrr}
    \hline
    \textbf{Category} & \textbf{Item} & \textbf{Q1} & \textbf{Q2} \\
    \hline
    Revenue & Product A & \$125K & \$142K \\
            & Product B & \$89K  & \$95K \\
    \hline
    Costs   & Marketing & \$42K  & \$38K \\
            & Operations & \$67K & \$71K \\
    \hline
    \textbf{Net} & & \textbf{\$105K} & \textbf{\$128K} \\
    \hline
  \end{tabular}

  \vspace{1em}

  \small Align numbers to the right, text to the left.
\end{frame}

\begin{frame}{Two-Column Layout}
  The theme supports standard Beamer columns:

  \vspace{1em}

  \begin{columns}[T]
    \begin{column}{0.48\textwidth}
      \textbf{Left Column}

      \begin{itemize}
        \item Clean layout
        \item Easy to read
        \item Balanced design
      \end{itemize}
    \end{column}

    \begin{column}{0.48\textwidth}
      \textbf{Right Column}

      Use columns to:
      \begin{enumerate}
        \item Compare concepts
        \item Show before/after
        \item Present alternatives
      \end{enumerate}
    \end{column}
  \end{columns}
\end{frame}

\subsection{Text Elements}

\begin{frame}{Quotations}
  \begin{quote}
    ``Less is more.''
  \end{quote}

  \vspace{0.5em}
  \hfill --- Ludwig Mies van der Rohe

  \vspace{2em}

  \begin{quote}
    ``Simplicity is the ultimate sophistication.''
  \end{quote}

  \vspace{0.5em}
  \hfill --- Leonardo da Vinci
\end{frame}

\subsection{Mathematics}

\begin{frame}{Mathematics}
  The Shibui theme handles mathematical content elegantly:

  \vspace{1em}

  \textbf{Inline math}: The quadratic formula is $x = \frac{-b \pm \sqrt{b^2-4ac}}{2a}$.

  \vspace{1em}

  \textbf{Display equations}:
  \begin{equation}
    \int_{-\infty}^{\infty} e^{-x^2} \, dx = \sqrt{\pi}
  \end{equation}

  \vspace{0.5em}

  \textbf{Aligned equations}:
  \begin{align}
    \nabla \times \mathbf{E} &= -\frac{\partial \mathbf{B}}{\partial t} \\
    \nabla \times \mathbf{B} &= \mu_0 \mathbf{J} + \mu_0 \epsilon_0 \frac{\partial \mathbf{E}}{\partial t}
  \end{align}
\end{frame}

\subsection{Code and Algorithms}

\begin{frame}[fragile]{Code Listings}
  Clean, readable code with syntax highlighting:

  \vspace{0.5em}

  \begin{lstlisting}[language=Python]
# Fibonacci sequence generator
def fibonacci(n):
    """Generate first n Fibonacci numbers."""
    a, b = 0, 1
    for _ in range(n):
        yield a
        a, b = b, a + b

# Usage example
for num in fibonacci(10):
    print(num, end=' ')
  \end{lstlisting}
\end{frame}

%--------------------------------------------------
\section{Data \& Figures}
%--------------------------------------------------

\subsection{Data Visualization Principles}

\begin{frame}{Presenting Data}
  When including figures or data visualizations:

  \begin{itemize}
    \item Keep charts simple and uncluttered
    \item Remove unnecessary gridlines and decorations
    \item Use the theme's color palette for consistency
    \item Ensure high contrast for readability
    \item Add concise, informative captions
  \end{itemize}

  \vspace{1em}

  \textbf{Remember}: The goal is to illuminate, not decorate.
\end{frame}

\begin{frame}{Visual Hierarchy}
  Good presentations guide the eye naturally:

  \vspace{1em}

  \begin{enumerate}
    \item \textbf{Most important}: Use size and weight
    \item \emph{Moderate importance}: Use italics or color
    \item Least important: Use smaller text or gray
  \end{enumerate}

  \vspace{1em}

  The Shibui theme's generous whitespace helps create natural visual hierarchy without competing elements.
\end{frame}

\subsection{Algorithms and Pseudocode}

\begin{frame}{Algorithms}

  \begin{algorithm}[H]
    \small
    \caption{Binary Search}
    \begin{algorithmic}[1]
      \Procedure{BinarySearch}{$A, n, T$}
        \State $L \gets 0$
        \State $R \gets n - 1$
        \While{$L \leq R$}
          \State $m \gets \lfloor (L + R)/2 \rfloor$
          \If{$A[m] < T$}
            \State $L \gets m + 1$
          \ElsIf{$A[m] > T$}
            \State $R \gets m - 1$
          \Else
            \State \Return $m$
          \EndIf
        \EndWhile
        \State \Return \textit{unsuccessful}
      \EndProcedure
    \end{algorithmic}
  \end{algorithm}
\end{frame}

\subsection{Graphics and Plots}

\begin{frame}{TikZ Diagrams}
  \centering
  \begin{tikzpicture}[scale=0.8]
    % Nodes
    \node[circle,draw=shibuitext,fill=shibuigray!20,text=shibuitext,minimum size=1cm] (A) at (0,0) {A};
    \node[circle,draw=shibuitext,fill=shibuigray!20,text=shibuitext,minimum size=1cm] (B) at (3,1) {B};
    \node[circle,draw=shibuitext,fill=shibuigray!20,text=shibuitext,minimum size=1cm] (C) at (3,-1) {C};
    \node[circle,draw=shibuiaccent,fill=shibuiaccent!15,text=shibuitext,minimum size=1cm] (D) at (6,0) {D};

    % Edges
    \draw[->,thick,color=shibuitext] (A) -- (B) node[midway,above,text=shibuitext] {5};
    \draw[->,thick,color=shibuitext] (A) -- (C) node[midway,below,text=shibuitext] {3};
    \draw[->,thick,color=shibuitext] (B) -- (D) node[midway,above,text=shibuitext] {2};
    \draw[->,thick,color=shibuitext] (C) -- (D) node[midway,below,text=shibuitext] {4};
    \draw[->,thick,color=shibuiaccent] (B) -- (C) node[midway,right,text=shibuiaccent] {1};
  \end{tikzpicture}

  \vspace{0.5em}

  \small Weighted directed graph using themed colors (shibuitext, shibuiaccent, shibuigray)
\end{frame}

\begin{frame}{PGFPlots: Standard Style}
  \centering
  \begin{tikzpicture}
    \begin{axis}[
      width=0.75\textwidth,
      height=0.8\textheight,
      xlabel={$x$},
      ylabel={$f(x)$},
      grid=major,
      grid style={color=shibuigray!30},
      legend pos=north east,
      axis line style={color=shibuitext},
      tick label style={color=shibuitext},
      label style={color=shibuitext}
    ]
      \addplot[color=shibuitext, thick, smooth, domain=-2:2] {x^2};
      \addlegendentry{$x^2$}

      \addplot[color=shibuiaccent, thick, smooth, domain=-2:2] {2^x};
      \addlegendentry{$2^x$}

      \addplot[color=shibuidarkgray, thick, dashed, domain=-2:2] {0.5*x};
      \addlegendentry{$0.5x$}
    \end{axis}
  \end{tikzpicture}
\end{frame}

\begin{frame}{PGFPlots: Shibui Style}
  Simple activation with \texttt{shibui style} (no grid, legend outside):

  \vspace{0.3em}

  \centering
  \begin{tikzpicture}
    \begin{axis}[
      shibui style,
      shibui colors,
      width=0.75\textwidth,
      height=0.8\textheight,
      xlabel={$x$},
      ylabel={$f(x)$}
    ]
      \addplot[smooth, domain=-2:2, samples=50] {x^2};
      \addlegendentry{$x^2$}

      \addplot[smooth, domain=-2:2, samples=50] {2^x};
      \addlegendentry{$2^x$}

      \addplot[smooth, domain=-2:2, samples=50] {0.5*x};
      \addlegendentry{$0.5x$}
    \end{axis}
  \end{tikzpicture}
\end{frame}

\begin{frame}{Dynamic Data Visualization}

  \vspace{0.3em}

  \centering
  \begin{tikzpicture}
    \begin{axis}[
      shibui style,
      width=0.75\textwidth,
      height=0.8\textheight,
      xlabel={Time ($t$)},
      ylabel={Signal Amplitude},
      ymin=-1.5, ymax=1.5
    ]
      % Simulated dynamic data
      \addplot[color=shibuitext, thick, smooth, domain=0:10, samples=100]
        {sin(deg(x))};
      \addlegendentry{Base signal}

      \addplot[color=shibuiaccent, thick, smooth, domain=0:10, samples=100]
        {sin(deg(x)) + 0.2*sin(deg(5*x))};
      \addlegendentry{With noise}
    \end{axis}
  \end{tikzpicture}
\end{frame}

%--------------------------------------------------
\section{Best Practices}
%--------------------------------------------------

\subsection{Design Principles}

\begin{frame}{Shibui Principles}
  Apply these principles when using this theme:

  \begin{itemize}
    \item \textbf{Show the content}: Let content take center stage
    \item \textbf{Reduce noise}: Every element should serve a purpose
    \item \textbf{Integrate text and graphics}: Create a unified whole
    \item \textbf{Respect your audience}: Clear, honest presentation
  \end{itemize}

  \vspace{1em}

  The theme provides the framework; you provide the clarity.
\end{frame}

\subsection{Technical Guidelines}

\begin{frame}{Technical Tips}
  \begin{itemize}
    \item Use \texttt{aspectratio=169} for widescreen (16:9)
    \item Use \texttt{aspectratio=43} for traditional (4:3)
    \item Uncomment \texttt{\textbackslash sectionframe} in theme for automatic section slides
    \item Keep slides focused---one main idea per slide
    \item Use generous spacing with \texttt{\textbackslash vspace}
  \end{itemize}
\end{frame}

%--------------------------------------------------
\section{Conclusion}
%--------------------------------------------------

\begin{frame}{Summary}
  The Shibui Beamer theme offers:

  \begin{itemize}
    \item \textbf{Elegant design} inspired by Japanese minimalism
    \item \textbf{Minimal interface} that stays out of your way
    \item \textbf{Clear navigation} through sections and slides
    \item \textbf{Multiple font options} including Fira Sans, Garamond, Charter, and more
    \item \textbf{Calm color palettes} including Solarized variants
  \end{itemize}

  \vspace{1em}

  Use it to create presentations that respect both content and audience.
\end{frame}

\begin{frame}{Thank You}
  \centering
  \vspace{2em}

  {\Large Questions?}

  \vspace{2em}

  \textcolor{shibuidarkgray}{%
    This theme is available at: \\
    \texttt{github.com/yourname/shibui-beamer}
  }
\end{frame}

%--------------------------------------------------
% APPENDICES
%--------------------------------------------------

\appendix

\section{Appendix}

\begin{frame}[allowframebreaks]{Theme Options Reference}
  Complete list of theme options:

  \vspace{0.3em}

  \begin{shibuiframe}{Color Themes}
    \texttt{light} -- Light cream background (default) \\
    \texttt{dark} -- Dark background \\
    \texttt{solarizedlight}, \texttt{solarizeddark}, \texttt{solarizedosaka} -- Solarized variants \\
    \texttt{nord} -- Arctic blue theme \\
    \texttt{gruvboxlight}, \texttt{gruvboxdark} -- Warm retro colors \\
    \texttt{autumn} -- Warm earthy tones
  \end{shibuiframe}

  \vspace{0.3em}

  \begin{shibuiframe}{Font Themes}
    \texttt{sans} -- Fira Sans (default) \\
    \texttt{serif} -- ET Book/Palatino \\
    \texttt{garamond} -- EB Garamond \\
    \texttt{charter} -- Charter \\
    \texttt{mono} -- Fira Mono
  \end{shibuiframe}

  \vspace{0.3em}

  \begin{shibuiframe}{Layout Options}
    \texttt{nosectionheader} -- Hide section name from header \\
    \texttt{navontop} -- Show navigation squares above title \\
    \texttt{nosub} -- Hide subsections in table of contents
  \end{shibuiframe}

  \vspace{0.3em}

  \begin{shibuiframe}{Progress Bar Options}
    \texttt{progressbar=basic} -- Single continuous bar (default) \\
    \texttt{progressbar=segmented} -- Section-based segments \\
    \texttt{progressbar=none} -- No progress bar, page numbers only
  \end{shibuiframe}

  \vspace{0.3em}

  \textbf{Usage:} \texttt{\textbackslash usetheme[nord,sans,navontop]\{Shibui\}} \\
  \textbf{Example:} \texttt{\textbackslash usetheme[progressbar=segmented]\{Shibui\}}
\end{frame}

\begin{frame}{Color Definitions}
  Theme colors available for use:

  \vspace{1em}

  \centering
  \begin{tabular}{ll}
    \hline
    \textbf{Color Name} & \textbf{Usage} \\
    \hline
    \texttt{shibuibg} & Background color \\
    \texttt{shibuitext} & Main text color \\
    \texttt{shibuiaccent} & Accent/highlight color \\
    \texttt{shibuigray} & UI elements (light) \\
    \texttt{shibuidarkgray} & Secondary text \\
    \texttt{shibuicodebg} & Code/box background \\
    \hline
  \end{tabular}

  \vspace{1em}

  \small Use with: \texttt{\textbackslash color\{shibuiaccent\}} or \texttt{\textbackslash textcolor\{shibuiaccent\}\{text\}}
\end{frame}

\begin{frame}{Custom Environments}
  Special environments provided by the theme:

  \vspace{0.5em}

  \begin{shibuiframe}{shibuiframe}
    Minimal box with background color for highlighting concepts.

    \textbf{Usage:}

    \texttt{\textbackslash begin\{shibuiframe\}\{Title\}} \\
    \texttt{~~Content...} \\
    \texttt{\textbackslash end\{shibuiframe\}}
  \end{shibuiframe}

  \vspace{0.5em}

  All standard Beamer environments work as expected: blocks, columns, overlays, etc.
\end{frame}

\begin{frame}{Custom Footer Text}
  Add custom text or logos above the progress bar:

  \vspace{0.5em}

  \begin{shibuiframe}{footertext Command}
    The \texttt{\textbackslash footertext\{\}} command lets you add custom text displayed above the progress bar in very small font.

    \vspace{0.3em}

    \textbf{Usage in preamble:}

    \texttt{\textbackslash footertext\{Conference 2024 -- Paris\}}

    \vspace{0.3em}

    Perfect for adding: conference names, dates, institutional logos, or copyright notices.
  \end{shibuiframe}

  \vspace{0.3em}

  The footer text appears on all slides except the title page and uses the secondary text color for minimal visual impact.
\end{frame}

\begin{frame}{Additional Resources}
  \begin{itemize}
    \item \textbf{Design References}
      \begin{itemize}
        \item Japanese aesthetics: Wabi-sabi, Shibui, Ma
        \item Minimalist design principles
        \item Typography in presentation design
      \end{itemize}

    \vspace{0.5em}

    \item \textbf{Color Schemes}
      \begin{itemize}
        \item Solarized color palette
        \item Nord theme
        \item Gruvbox retro colors
      \end{itemize}

    \vspace{0.5em}

    \item \textbf{Beamer Documentation}
      \begin{itemize}
        \item Official Beamer user guide
        \item Theme development guide
      \end{itemize}
  \end{itemize}
\end{frame}

\begin{frame}[allowframebreaks]{Troubleshooting}
  \begin{shibuiframe}{Font Issues}
    If ET Book font is not available, Palatino is used as fallback. For Fira Sans, ensure the package is installed: \texttt{texlive-fonts-extra}
  \end{shibuiframe}

  \vspace{0.5em}

  \begin{shibuiframe}{Color Problems}
    Ensure you're using XeLaTeX or pdfLaTeX with appropriate color packages. Theme loads colors automatically.
  \end{shibuiframe}

  \vspace{0.5em}

  \begin{shibuiframe}{Progress Bar}
    Progress bar is automatically hidden on title page and plain frames. Three modes available:

    \begin{itemize}
      \item \textbf{basic} (default): Single continuous bar
      \item \textbf{segmented}: Section-based segments with gaps. Past sections are dimmed, current section fills progressively
      \item \textbf{none}: No bar, only page numbers
    \end{itemize}

    Use \texttt{noframenumbering} option to exclude specific frames from count.
  \end{shibuiframe}
\end{frame}

\end{document}
